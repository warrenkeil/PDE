\documentclass{article}
\usepackage{pgf,tikz,tikzscale} 
\usepackage{amssymb}
\usepackage{tcolorbox}
\usepackage{xcolor}
\usepackage[utf8]{inputenc}
\usepackage[english]{babel}
\usepackage{multicol}
\usepackage{enumerate}	
\usepackage{graphicx,lipsum,pgfplots} 
\usepackage{amsmath, amsthm}                 
\usepackage[top=1in,bottom=1in, left=1in, right=1in] {geometry}  
\usepackage{fancyhdr}       



\pagestyle{fancy}              
\lhead{Math 5910 \newline HW: 3.1-3.3}   
\rhead{Warren Keil}







\begin{document}
\setlength{\parindent}{0cm}   %%%%%%%% KEEP THIS  for block style para. 



%%%%%%%%%%%%%%%%%     3.1     1   %%%%%%%%%%%%%%%%%%%%
\textbf{Section 3.1}

\textbf{1.} Consider the initial boundary value problem for the wave equation
\begin{align}
u_{tt} &= c^2u_{xx}, \hspace{4mm} 0<x<\pi, \hspace{4mm} t>0 \\
u(0,t) &= u(\pi,t)=0, \hspace{4mm} t>0 \\
u(x,0) &= f(x), \hspace{4mm} u_t(x,0)=0, \hspace{4mm} 0<x<\pi,
\end{align}
on a bounded spatial domain. Use the fundamental set of solutions
\[
 u_n(x,t) = \cos nct \sin nx,  n=1,2,\ldots,
\]
which satisfy (1) and (2), to determine a formal solution of (1)-(3). Also find the solution if the initial conditions are changed to 
\[
u(x,0)=0, \hspace{4mm} u_t(x,0)=g(x), \hspace{4mm}0 <x<\pi.
\]
Observe that these calculations amount to resolving an arbitrary wave into its fundamental modes. 


\vspace{3mm}
\textit{Solution.} First, we verify that the "fundamental set of solutions" hold for equations (1) and (2). Let \(u_n(x,t)  
= \cos nct \sin nx,  \hspace{3mm} n\in\mathbb{N} \). Notice:
\begin{align*}
u_{nt}&= -nc \sin nct \sin nx  &   u_{nx}&=  -n \cos nct \cos nx \\
 &  	       &       &   \\
  u_{ntt}&= -n^2c^2 \cos nct \sin nx 	       &      u_{nxx} &=  -n^2 \cos nct \sin nx 
\end{align*}
Thus we see that \(u_{tt}= -n^2c^2 \cos nct \sin nx \) and \(c^2 u_{xx} =  -c^2n^2 \cos nct \sin nx \Rightarrow u_{tt}=c^2u_{xx}\). Next we observe the boundary condition
\[u(0,t) = \cos nct \sin n\cdot 0 = 0 \text{  and  } u(\pi,t) = \cos nct \sin n \pi = 0 . 
\]
Thus both equations (1) and (2) are satisfied. Next we use the same arguments that Fourier made, that the solution of \(u\) can be constructed from an infinite sum of our fundamental set of solutions
\[ u(x,t) = \sum_{n=1}^{\infty}  a_n \cos nct \sin nx. 
\]
Applying the boundary condition we solve for  \(a_n\),
\begin{align*}
u(0,t) = f(x) &= \sum_{n=1}^{\infty}  a_n  \sin nx \\
f(x) \sin(kx) &= \sum_{n=1}^{\infty}  a_n  \sin nx \sin(kx)  & \text{ multiply both sides by } \sin(kx) dx\text{ for some fixed }k \\
\int_0^\pi f(x) \sin(kx) dx &= \int_0^\pi \sum_{n=1}^{\infty}  a_n  \sin nx \sin(kx)  dx& \text{ integrate both sides}\\
\int_0^\pi f(x) \sin(kx) dx &= a_k \int_0^\pi \sin^2(kx) dx& \text{ since all other terms = zero when \(n\neq k\)}\\
\int_0^\pi f(x) \sin(kx) dx &= a_n/2 \int_0^\pi 1-\cos (2kx )dx \\
\int_0^\pi f(x) \sin(kx) dx &= \frac{\pi}{2} a_n \\
\Rightarrow a_n &= \frac{2}{\pi}\int_0^\pi f(x) \sin(kx) dx.
\end{align*}

Thus the solution of \(u\) is 
\begin{align*}
u(x,t) &= \sum_{n=1}^{\infty}  a_n \cos nct \sin nx \\
a_n &= \frac{2}{\pi}\int_0^\pi f(x) \sin(kx) dx.
\end{align*}
Lastly we check to make sure the initial condition of \(u_t\) is met. 
\[
u_t(x,0) =  \sum_{n=1}^{\infty}  a_n \cdot nc \sin0 \sin nx  =\sum_{n=1}^{\infty}0=0.
\]
\begin{flushright}
\( \diamondsuit \)
\end{flushright}
If the initial conditions are changed to 
\[
u(x,0)=0, \hspace{4mm} u_t(x,0)=g(x), \hspace{4mm}0 <x<\pi
\]
then we now proceed by using the same steps above but we will solve for \(a_n\) with the initial condition of \(u_t\). Only this time we find if we use the same initial 'solution,' it does not work. So we guess a different solution \(u(x,t) = \sum_{n=1}^{\infty}  a_n \sin nct \sin nx\). So first we observe that the initial condition is satisfied since, \(u(x,0) =  \sum_{n=1}^{\infty}  a_n \sin 0 \sin nx = 0\). To solve for \(a_n\) we get
\begin{align*}
u_t(x,0) = g(x) &= \frac{\partial}{\partial t}  \sum_{n=1}^{\infty}  a_n \sin nct \sin nx \Big|_{t=0}  \\
g(x)&= \sum_{n=1}^{\infty} nc \cdot a_n \cos 0 \sin nx         &       \text{mult by \(\sin mx\) for some m}    &   \\
\int_0^{\pi} g(x)\sin mx dx &= \int_0^{\pi} \sum_{n=1}^{\infty} nc \cdot a_n  \sin nx \sin mx   dx   & \text{integrate both side w.r.t. x}         &   \\
\int_0^{\pi} g(x) \sin mx dx &= \sum_{n=1}^{\infty}  a_n\int_0^{\pi}  nc \sin nx \sin mx  dx       &   \text{since } a_n \text{ not dependent on } x  &   \\
\int_0^{\pi} g(x) \sin nx dx &=  a_n\int_0^{\pi}  nc \sin^2 nx   dx  &          \text{ since LHS =0 whenever } m\neq n&   \\
\int_0^{\pi} g(x) \sin nx dx&=  a_n \frac{\pi}{2} nc  &        \text{ integration by Mathematica}   &   \\
a_n &= \frac{2}{\pi n c }\int_0^{\pi} g(x) \sin nx dx .      &          &  
\end{align*}
\begin{flushright}
\( \qed\)
\end{flushright}



\newpage
\textbf{Section 3.2}

%%%%%%%%%%%%%%%%%%%%%%%%%%%%%%%%%%%%%%%%     3.2       3
\textbf{3.} The functions \(1, x, x^2, x^3 \) are independent functions on the interval \( [-1,1] \). 


a) Use the preceding exercise to generate a set of four orthogonal polynomials \(P_0(x), \ldots, P_3(x)\) on \( [-1, 1]\), called the \textbf{Legendre polynomials} .

\vspace{4mm} 
\textit{Solution.} We use the Gramm Schmidt algorithm to generate the set of orthogonal functions, the Legendre polynomials. We will use the same notation as in the text with \(f_1=1, f_2=x, f_3=x^2, \text{ and } f_4=x^3\). Then by the Gramm Schmidt algorithm, let \(g_1=1\),
\begin{align*}
g_2&=f_2-\frac{(f_2,g_1)}{ \|g_1\|^2}g1,     &    g_3&=f_3 -\frac{(f_3,g_2)}{ \|g_2\|^2}g_2 - \frac{(f_3,g_1)}{ \|g_1\|^2}g_1   \\ 
&= x-\frac{\int_{-1}^1 x dx}{ \int_{-1}^1 dx}     &       &= x^2-\frac{\int_{-1}^{1} x^2\cdot x dx}{ \int_{-1}^{1} x\cdot x dx}x  - \frac{\int_{-1}^1 x^2 dx}{\int_{-1}^1dx}\\
&=x- \frac{0}{2}        &        &= x^2 -\frac{\int_{-1}^1 x^3 dx}{\int_{-1}^1 x^2 dx}x - \frac{2/3}{2}\\
&=x        &        &=x^2 - \frac{0}{2/3}x- \frac{2}{6}   \\
&       &        &=x^2-\frac{1}{3}   \\
g4&= f_4- \frac{ (f_4,g_3 )}{\| g_3 \|^2}g_3 - \frac{ (f_4,g_ 2)}{\| g_2 \|^2}g_2 - \frac{ (f_4,g_1 )}{\| g_1 \|^2}g_1      &        &  \\
&= x^3- \frac{ \int_{-1}^1 x^3(x^2-1/3) dx }{\int_{-1}^1  x^6 dx}(x^2-1/3)- \frac{ \int_{-1}^1 x^3(x) dx}{\int_{-1}^1 x^2  dx}x-\frac{ \int_{-1}^1 x^3dx }{\int_{-1}^1 dx  } &        &   \\
&=x^3- \frac{ \int_{-1}^1 x^5-1/3x^3 dx }{\int_{-1}^1  x^6 dx}(x^2-1/3)- \frac{ \int_{-1}^1 x^4 dx}{\int_{-1}^1 x^2  dx}x-\frac{ \int_{-1}^1 x^3dx }{\int_{-1}^1 dx  }        &        &   \\
&=  x^3- \frac{  0-(1/3)\cdot 0 }{2/7}(x^2-1/3)- \frac{ 2/5}{2/3}x-\frac{ 0}{2 }         &        &   \\
&=  x^3-\frac{3}{5 } x     &        &  
\end{align*}
Thus have found an orthogonal set of polynomial functions. These are also known as the first four Legendre polynomials denoted:
\[ P_0(x)=1, \hspace{3mm} P_1(x)=x, \hspace{3mm} P_2(x)=x^2 -\frac{1}{3}, \hspace{3mm} P_3(x)=x^3-\frac{3}{5}x. \]

\newpage

b) Find the best approximation of \(e^x\) on \([-1,1]\) of the form
\[
e^x \approx c_0P_0(x) + c_1P_1(x) + c_2P_2(x) + c_3P_3(x). 
\] 
\textit{Solution.}  Since we have found an orthogonal set of functions as guaranteed by the Gramm Schmidt algorithm, Then by theorem 3.6,  we can write \(e^x = \sum_{n=0}^\infty c_n f_n(x) \) where \(f_n = P_n\) and 
\[ c_n=\frac{1}{\|f_n\|^2 }(f,f_n), \hspace{3mm} n=0,1,2,\ldots \]
Thus we find,
\begin{align*}
c_0 &= \frac{ \int_{-1}^1 e^x dx}{\int_{-1}^1  dx }   &   c_1&=\frac{ \int_{-1}^1 e^x x dx}{\int_{-1}^1 x^2  dx }    \\
&= \frac12 (e-\frac1e) &  & \text{ (we use mathematica here to be more efficient on time)}  \\
&\approx 1.1752 &  &\approx 1.10364 \\ 
&  & & \\
c_2 &=\frac{ \int_{-1}^1 e^x(x^2-1/3) dx}{\int_{-1}^1(x^2-1/3)^2  dx }     &     c_3 &=\frac{ \int_{-1}^1 e^x(x^3-3x/5) dx}{\int_{-1}^1(x^3-3x/5)^2  dx }   \\
&=\frac{15 \left(e^2-7\right)}{4 e}  &    &=\frac{175}{8} \left(\frac{74}{5 e}-2 e\right)       \\
&\approx 0.536722 & &\approx 0.176139
\end{align*}








\vspace{4mm} 
c) Plot \(e^x\) and the approximation on a set of coordinate axes. 

\vspace{4mm} 
\textit{Solution.} We use mathematica to make the following graph. We will post the code at the end of this paper. Notice the approximation function is in blue and the graph of \(e^x\) is in orange. We are also aware that it is hard to decipher the two different lines on this graph. The difference will become more apparent when we plot the error. Our approximation is 
\[
0.176 \left(x^3-\frac{3 x}{5}\right)+0.537 \left(x^2-\frac{1}{3}\right)+1.104 x+1.175
\]

\includegraphics[scale=.6]{Legendre.png}


\newpage

d) What is the pointwise error? What is the maximum pointwise error over \([-1,1]\)? What is the mean-square error?

\vspace{4mm} 
\textit{Solution.} The pointwise error is given by the function of page 136,
\[
E_n(x) = f(x) - \sum_{n=1}^Nc_nf_n(x).
\]
We use mathematica to quickly compute this error. 
\begin{verbatim}
g2[x_] := 1.175*1 + 1.104*x + .537*(x^2 - (1/3)) + .176*(x^3 - (3*x/5)) 
f[x_] := Exp[x]
E4[x_] := f[x] - g2[x]
Plot[E4[x], {x, -1, 1}]
\end{verbatim}
\includegraphics[scale=.6]{pointw}

To find the maximum pointwise error, we run the following code,
\begin{verbatim}
FindMaximum[{E4[x], -1 < x < 1} , {x}]
\end{verbatim}
To find that the error has a maximum value on \([-1,1]\) of \(.0108818\) when \(x=1\). 


The mean-square error is given by,
\begin{align*}
e_4 &= \int_a^b |f(x)-\sum_{n=1}^4 c_nf_n(x)|^2 dx \\
&= \int_a^b | e^x - 0.176 \left(x^3-\frac{3 x}{5}\right)+0.537 \left(x^2-\frac{1}{3}\right)+1.104 x+1.175 |^2 dx
\end{align*}
\begin{verbatim}
meansquare = Integrate[ Abs[E4 - g2[x]]^2 , {x, -1, 1} ]
1.1428571428571429`*^-8 (3.17316619`*^8 + 1.75`*^8 Im[E4]^2 - 
   4.1125`*^8 Re[E4] + 1.75`*^8 Re[E4]^2)
\end{verbatim}
Thus, we have found our mean squared error to be  \( 1.1428571428571429 \cdot 10^{-8} \)



\newpage
%%%%%%%%%%%%%%%%%%%%%%%%%%%%%%%%%%%%%%%%     3.2       5
\textbf{5.} Let \(f\) be defined and integrable on \([0,l]\). The orthogonal expansion \[
\sum_{n=1}^{\infty} b_n \sin\frac{n\pi x}{l}, b_n = \frac{2}{l} \int_0^l f(x) \sin\frac{n\pi x}{l} dx, 
\]
is called the \textbf{Fourier sine series} for \(f\) on \([0,l]\). Find the Fourier sine series for \(f(x)=\cos x \) on \( [0,2\pi]\) and plot  a 6-term approximation. What is the Fourier sine series of \(f(x) = \sin x \) on \( [0,\pi]\)?


\vspace{4mm} 
\textit{Solution.} To calculate the \(b_n\) terms we run the following mathematica code. We notice that the \(b_n\) terms are zero for every odd \(n\). So we convert to new function of \(n\) to throw out the odd terms. 

\begin{verbatim}
b[n_] := (2/Pi)*Integrate[Cos[x]*Sin[(n*Pi*x)/Pi] , {x, 0, Pi}]
b2[n_] := b[2*n]
{b[2], b[4], b[6], b[8], b[10], b[12]}
{8/(3 \[Pi]), 16/(15 \[Pi]), 24/(35 \[Pi]), 32/(63 \[Pi]), 40/(
 99 \[Pi]), 48/(143 \[Pi])}
 Plot[h[x], {x, 0, Pi}]
\end{verbatim}
\[
h(\text{x$\_$})\text{:=}\frac{8 \sin (2 x)}{3 \pi }+\frac{16 \sin (4 x)}{15 \pi }+\frac{24 \sin (6 x)}{35 \pi }+\frac{32 \sin (8 x)}{63 \pi }+\frac{40 \sin (10 x)}{99 \pi }+\frac{48 \sin (12 x)}{143 \pi }\]

\includegraphics[scale=.6]{cosfour}


To calculate the Fourier sine series for \(\sin x\) we do:
\begin{align*}
b_n&= 2/\pi \int_0^\pi \sin x \sin nx dx \\
&= \frac{\cos (x) \sin (n x)-n \sin (x) \cos (n x)}{n^2-1}
\Big|_0^{\pi}  \hspace{6mm} \text{from mathematica. We solve for \(n>1\) first.} \\
&= \frac{0-0}{n^2-1}   = 0 \hspace{6mm} \text{for only when \(n>1\). }
\end{align*}
Now we solve for the case when \(n=1\) separately. Let \(n=1\). 
\begin{align*}
b_1&= 2/\pi \int_0^\pi \sin x \sin x dx \\
&= \frac{2}{\pi} (\frac{x}{2}-\frac{1}{4} \sin (2 x)\Big|_0^\pi) \\
&= \frac{2}{\pi} (\frac{\pi}{2}-0-(0-0))\\
&=1
\end{align*}
Thus, the Fourier sine series for \(\sin x\) is 
\[ \sin x = \sum_{n=1}^\infty b_n \sin nx = 1\cdot \sin x + 0 + 0+ \ldots  = \sin x \]

\newpage

%%%%%%%%%%%%%%%%%%%%%%%%%%%%%%%%%%%%%%%%     3.2       8
\textbf{8.} For \(f,g\in L^2[a,b]\), prove the \textbf{Cauchy-Schwarz inequality}
\[
| (f,g) | \leq \|f\| \|g\|
\]

\vspace{4mm} 
\textit{Solution.} We first use the hint and let \(q(t) = (f + tg, f+tg),  \forall t \in \mathbb{R}\).  Next, we observe that 
\begin{align*}
q(t) = \langle f + tg, f+tg \rangle &= \langle f ,f\rangle + 2t \langle f,g\rangle + t^2 \langle g,g\rangle \\
&= \|f \|^2 +2t\langle f,g\rangle + t^2\|g\|^2\\
&= \|g\| t^2 + 2\langle f,g\rangle t + \|f \|^2
\end{align*}
Now, since \(q(t)= (f + tg, f+tg) = \|f+tg\|^t \Rightarrow q(t) \geq 0\). We have shown \(q(t)\) is a quadatic polynomial of \(t\) and that it is non-negative, thus it can have at most one root (with multiplicity = 2 if it exists). Therefore, this means when solving the quadratic equation, the term \(\sqrt{b^2-4ac} \leq 0 \). This is because if  \( \sqrt{b^2-4ac} >0 \) then the function \(q(t)\) would have exactly two real roots which would imply that it has values for which it is negative. But since it is equivalent to the norm squared of \(f+tg\), then this cannot ever happen \(\rightarrow\!\leftarrow\). Thus, 
\begin{align*}
& & \sqrt{b^2-4ac} &\leq 0 \\
\iff&& b^2-4ac &\leq 0\\ 
\iff && (2\langle f,g\rangle)^2 - 4 \|g\|^2 \|f \|^2  &\leq 0 \\ 
\iff &&   4 \langle f,g\rangle^2  &\leq 4(\|g\| \|f \|)^2  \\
\iff &&   4/4 \sqrt{\langle f,g\rangle^2 }  &\leq 4/4 \sqrt{(\|g\| \|f \|)^2 } \\
\iff && | \langle f,g\rangle | &\leq \|g\| \|f \| 
\end{align*}
\begin{flushright}
 \( \blacksquare \) 
\end{flushright}
note: we need an absolute value sign on left side since inner products can be negative. Norms are non-negative, thus we do not need to specify absolute values on right side. 
\newpage
\textbf{Section 3.3}

%%%%%%%%%%%%%%%%%%%%%%%%%%%%%%%%%%%%%%%%     3.3       1
\textbf{1.} Find the Fourier series for the \(2\pi\)-periodic square wave shown in Figure 3.7. Sketch a two-term, a four-term and a six-term approximation. 

\vspace{4mm} 
\textit{Solution.} We first observe that the squarewave shown in figure 3.7 is symmetric about the y-axis, therefore it is even. Thus the \(b_n\) sine terms equal zero for all \(n\). We next compute the \(a_n\) with the comment that we will assume we are starting at \(n=1\) if any problems arise with \(a_0\). Next we observe that the squarewave equals 1 between \(-\pi/2 , \pi/2\) and also for every other interval \(\pi\) distance apart. Thus we will make our interval be \( -\pi,\pi\). Thus, the coefficients are
 \begin{align*}
a_n &= \frac{1}{\pi} \int_{-\pi}^{\pi} f(x) \cos nx dx \\
&= \frac{1}{\pi}\left[ \int_{-\pi}^{-\pi/2} 0 dx + \int_{-\pi/2}^{\pi/2} \cos nx dx +  \int_{\pi/2}^{\pi} 0 dx\right] \\
&=  \frac{1}{\pi}\left[ \int_{-\pi/2}^{\pi/2} \cos nx dx  \right] \\
&=  \frac{1}{\pi} (\frac1n \sin nx ) \Big|_{-\pi/2}^{\pi/2} \\
&=  \frac{1}{n\pi} (\sin(\frac{n\pi}{2}) - \sin(\frac{-n\pi}{2} ) ) \\
&=  \frac{1}{n\pi} (\sin(\frac{n\pi}{2}) + \sin(\frac{n\pi}{2} ) )   &\text{since sine is odd }\\
&=  \frac{2}{n\pi} (\sin(\frac{n\pi}{2}))  & \text{ for } n=1,2,3,\ldots
\end{align*}
Since we had to divide by \(n\), then we will compute \(a_0\) separately. 
\begin{align*}
a_0 &=  \frac{1}{\pi} \int_{-\pi/2}^{\pi/2} \cos 0 dx \\
&=\frac{1}{\pi} \int_{-\pi/2}^{\pi/2} dx\\
&=\frac{1}{\pi} ( \pi/2 - (-\pi/2))\\
&= \frac{1}{\pi} (\pi) \\
&= 1
\end{align*}
Now we have all of the \(a_n\) coefficients, we have the Fourier series is
\[
F(x) = \frac12 + \sum_{n=1}^{\infty} \frac{2}{n\pi} (\sin(\frac{n\pi}{2}))\cos nx 
\]

\newpage
We use mathematica to create and plot 2-term, 4-term, and 6 term approximations. 

\begin{verbatim}
F[x_, n_] := .5 + Sum[(2/i*Pi)*Sin[i*Pi/2]*Cos[i*x], {i, n}]
Plot[{ F[x, 2], F[x, 4], F[x, 6]}, {x, -4, 4} ]
\end{verbatim}
\includegraphics{squarew}


The blue line is the two term function. The orange line is the four term function. The green line is the six term function. 

\newpage


%%%%%%%%%%%%%%%%%%%%%%%%%%%%%%%%%%%%%%%%     3.3       5
\textbf{5.} Let \( f(x) = -\frac{1}{2} \) on \(- \pi < x \leq 0 \) and \( f(x) = \frac{1}{2} \) on \(0 \leq x \leq \pi \).  Show that the Fourier series for \(f\) is
\[
\sum_{n=1}^{\infty} \frac{2}{(2n-1)\pi } \sin(2n-1) x. 
\]
Sketch the graph of \(s_1(x),s_3(x), s_7(x) \) and \(s_10(x)\) and compare with \(f(x)\). Observe the Gibbs phenomenon.


\vspace{4mm} 
\textit{Solution.} First, observe that the function described is odd. So we will only have to us the \(b_n\) sine terms. Also given in the problem is that \(l=\pi\). So, 

\begin{align*}
b_n &= \frac l\pi \left( \int_{-\pi}^0 -\sin nx dx + \int_0^\pi \sin nx dx \right) \\
&=  \frac 1\pi \left(1/n  \cos nx \Big|_{-\pi}^0  - (1/n \cos nx ) \Big|_0^{\pi}  \right) \\
&=  \frac 1\pi \left(1/n  \cos nx \Big|_{-\pi}^0  + (1/n \cos nx ) \Big|_{\pi}^0 \right) \\
&=  \frac 1\pi  1/n( \cos 0 - \cos(-n\pi) ) + 1/n(\cos 0 - \cos(n\pi) )  \\
&= \frac 1{n\pi}[ (1- \cos(n\pi) ) + (1-\cos(n\pi) ) ] \\
&=\frac 2{n\pi}[ (1- (-1)^n ] 
\end{align*}
When  \(n\) is even, then \(b_n=0\). 
So force the \(n\) terms to be odd by only defining the series for \(2n-1\). Then when \(n=1,2,3,\ldots\),  then \(2n-1= 1,3,5,7,\ldots. \). So let \(m = 2n-1\). Then  \(b_m=\frac{2}{m\pi}\) And furthermore, the Fourier sine series is defined \textit{precisely} for these \(b_m\)s. 

Then it follows that the Fourier series for the function described is 
\[
F(x) = \sum_{m=1}^\infty \frac 2{m\pi} \sin mx = \sum_{n=1}^\infty \frac 2{(2n-1)\pi} \sin(2n-1)x
\]
\begin{flushright}
\( \qed\)
\end{flushright}









\end{document}






